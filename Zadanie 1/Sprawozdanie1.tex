\documentclass[polish,polish,a4paper]{article}
\usepackage[T1]{fontenc}
\usepackage[utf8]{inputenc}
\usepackage{pslatex}
\usepackage{pgfplots}
\usepackage{circuitikz} 
\usepackage{setspace}
\usepackage{caption}
\usepackage{amssymb}
\usepackage{amsmath}
%\usetikzlibrary{circuits.ee.IEC}
\usepackage{anysize}
\usepackage{graphicx}
\usepackage{hyperref}
\usepackage{float}
\usepackage[polish]{babel}
\hypersetup{
	colorlinks=true,
	linkcolor=blue,
	filecolor=magenta,      
	urlcolor=cyan,
}
\urlstyle{same}

\marginsize{2.5cm}{2.5cm}{2cm}{2cm}

\title{}

\author{}

\begin{document}
	
	\begin{titlepage}
			\vspace*{\fill}
				    \begin{center}
				    {\Large Fizyka dla informatyków \\[0.1cm]
				    	Sprawozdanie z zadania w zespołach nr. 1\\[0.1cm]
			    		prowadzący: dr. Gustaw Szawioła\\[0.7cm]}
					{\huge Zależność drgań oscylatora harmonicznego z siłą wymuszającą od częstości $\omega$ siły wymuszjącej. - Eksperyment numeryczny}\\ [0.7cm]
					{\large autorzy:}\\[0.1cm]
					{\Large Mariusz Sałaj (136795), Rafał Wójcik (136831),Piotr Więtczak(132339),\\ [0.1cm]
						 Robert Ciemny(136693), Kamil Basiukajc(136681)}\\[0.5cm]
					\today
				\end{center}
			\vspace*{\fill}
	\end{titlepage}
	

			\section{Cel zadania}
		Celem tego zadania jest, kozrzystając z programu {\em Mathematica} dostępnego w chmurze, zbadanie na drodze ekseprymentu numerycznego zależności drgań oscylatora harmonicznego z siłą wymuszającą $\dfrac{d^{2}x(t)}{dt^{2}} + b \dfrac{dx}{dt} +
		\omega_{0}^2 x(t) = sin(\omega t)$ od częsości $ \omega $ siły wymuszającej. Należy wykonać wykres zależności amplitudy drgań w funkcji częstości $ \omega $ i wyznaczyć tzw. częstość razonansową, dla której drgania 
		przyjmują wartość największą. Do obliczeń przyjmujemy $f=1$, a reszta wartości według wskazań prowadzącego.
		
		\section{Wyznaczenie wartości: $s$, $\triangle s$, $w_{0}$, $b$, według wskazań prowadzącego}
		\begin{gather*}
		s = \frac{1}{N}\sum_{i=1}^{N} nralbumu_{i}\\
		s = \dfrac{1}{5} \cdot 679339 = 135867,8\\
		\triangle s = \sqrt{\frac{1}{N}\sum_{i=1}^{N} (nralbumu_{i} -s)^2}\\
		\triangle s = \sqrt{\frac{1}{5} \cdot 15582132,8} = 1765,340352\\
		\omega_{0} = \triangle s + 1\\
		\omega_{0} = 1766,340352\\
		b = \frac{1}{4}\omega_{0}\\
		b= 441,5850881\\
		\end{gather*}
		
		\section{Przeprowadzenie eksperymentu numerycznego}
		
		
		\section{Przedstawienie przykładowych rozwiązań numerycznych $x(t)$ przy warunku początkowym $ x(0) = 0  $ oraz $  v(0)= 0 $ w przedziale czasu $ 0 \leq t \leq n\frac{2\pi}{\omega_{0}}$, $ n = JANPAWEŁ $ dla przypadków:}
		\subsection{$\omega = \sqrt{\omega_{0}^2 - \frac{1}{2}b^2}$}
		\subsection{$\omega = \frac{3}{4}\sqrt{\omega_{0}^2 - \frac{1}{2}b^2}$}
		\subsection{$\omega = \frac{5}{4}\sqrt{\omega_{0}^2 - \frac{1}{2}b^2}$}

	
	
	
	\newpage

\end{document}


